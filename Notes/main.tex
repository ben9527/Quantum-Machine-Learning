\documentclass{book}
\usepackage{subfiles}
\usepackage[toc,page]{appendix}

\usepackage{tikz}
\usetikzlibrary{trees}
\tikzset{
  invisible/.style={opacity=0},
  visible on/.style={alt={#1{}{invisible}}},
  alt/.code args={<#1>#2#3}{%
    \alt<#1>{\pgfkeysalso{#2}}{\pgfkeysalso{#3}} % \pgfkeysalso doesn't change the path
  },
  properties/.style={green, ultra thick},
}

\oddsidemargin=17pt \evensidemargin=17pt
\headheight=9pt     \topmargin=26pt
\textheight=564pt   \textwidth=433.8pt
\date{}
\usepackage{url}
\usepackage{amsmath}
\usepackage{amsfonts,amssymb,amsthm,float,graphicx}
\usepackage{physics}
\usepackage{graphicx}
\usepackage{mathtools}
\usepackage{float}
\usepackage{hyperref}
\hypersetup{
    colorlinks=true, %set true if you want colored links
    linktoc=all,     %set to all if you want both sections and subsections linked
    linkcolor=blue,  %choose some color if you want links to stand out
}
\usepackage[backend=biber]{biblatex}
\addbibresource{course_notes.bib}

%new math symbols taking no arguments
\newcommand\0{\mathbf{0}}
\newcommand\CC{\mathbb{C}}
\newcommand\FF{\mathbb{F}}
\newcommand\NN{\mathbb{N}}
\newcommand\QQ{\mathbb{Q}}
\newcommand\RR{\mathbb{R}}
\newcommand\ZZ{\mathbb{Z}}
\newcommand\bb{\mathbf{b}}
\newcommand\kk{\Bbbk}
\newcommand\mm{\mathfrak{m}}
\newcommand\pp{\mathfrak{p}}
\newcommand\xx{\mathbf{x}}
\newcommand\yy{\mathbf{y}}
\newcommand\GL{\mathit{GL}}
\newcommand\into{\hookrightarrow}
\newcommand\nsub{\trianglelefteq}
\newcommand\onto{\twoheadrightarrow}
\newcommand\minus{\smallsetminus}
\newcommand\goesto{\rightsquigarrow}
\newcommand\nsubneq{\vartriangleleft}

%redefined math symbols taking no arguments
\newcommand\<{\langle}
\renewcommand\>{\rangle}
\renewcommand\iff{\Leftrightarrow}
\renewcommand\phi{\varphi}
\renewcommand\implies{\Rightarrow}

%new math symbols taking arguments
\newcommand\ol[1]{{\overline{#1}}}

%redefined math symbols taking arguments
\renewcommand\mod[1]{\ (\mathrm{mod}\ #1)}

%roman font math operators
\DeclareMathOperator\aut{Aut}

%for easy 2 x 2 matrices
\newcommand\twobytwo[1]{\left[\begin{array}{@{}cc@{}}#1\end{array}\right]}

%for easy column vectors of size 2
\newcommand\tworow[1]{\left[\begin{array}{@{}c@{}}#1\end{array}\right]}

\newtheorem{theorem}{Theorem}[section]
\newtheorem{corollary}{Corollary}[theorem]
\newtheorem{lemma}[theorem]{Lemma}
\newtheorem{proposition}[theorem]{Proposition}
\newtheorem{exercise}[theorem]{Exercise}
\newtheorem{definition}[theorem]{Definition}
\newtheorem{fact}[theorem]{Fact}
\newtheorem{algorithm}[theorem]{Algorithm}
\newtheorem{example}[theorem]{Example}

\title{Quantum Algorithms and Learning Theory}
\author{Faris Sbahi}

\begin{document}
\maketitle

\chapter*{Abstract}

In this document, we provide a presentation of the latest results in quantum learning theory alongside theoretical extensions. We also provide experimental analyses of quantum feature maps which can be used for supervised learning. 

The first part of our paper is a review: First, we present an overview of quantum computation and information. Next, we present a review of the brief history of quantum machine learning. The subsequent part of our paper is an analysis of recent results in quantum learning theory: (1) information theoretic bounds on quantum computation learning, (2) supervised learning using hybrid quantum-classical circuits, and (3) Tang's \cite{tang2018quantum} idea of least-square sampling providing parallel classical algorithms for quantum machine learning algorithms that solve singular value transformation problems.

The last part of our paper provides new results on quantum feature maps which seek to solve the quantum encoding problem by encoding data inputs into a quantum state that implicitly performs the feature map given by a kernel function. Therefore, if the kernel is sufficiently difficult to evaluate classically, then there may exist a quantum advantage. Hence, we provide a geometric analysis of the properties of a kernel that may provide quantum advantage, and provide experimental results to demonstrate the robustness of particular candidate maps.

\tableofcontents

\chapter{Introduction}

Rough draft readers:

The "Introduction" and "Preliminaries" chapters are in progress. I plan a standard review of the necessary background from quantum information theory (see \cite{nielsen2010quantum} and \cite{wilde2013quantum}) to make the essential chapters of this thesis interpretable to a general Physics audience. Of course, I will primarily restate theorems and provide references in order to keep this portion succinct. For the time being, I've included my personal notes that I've kept since I began working on this project.

\subfile{nielsen_chuang_notes.tex}

%\subfile{crypto_notes.tex}

\subfile{quantum_learning_notes.tex}

\begin{appendices}
\section{Quantum Mechanics}

\begin{definition}
\label{pauli}
Pauli Matrices

\begin{align*}
\sigma_x &= X = \begin{pmatrix} 0 & 1 \\ 1 & 0\end{pmatrix} \\
\sigma_y &= Y = \begin{pmatrix} 0 & -i \\ i & 0\end{pmatrix}\\
\sigma_z &= Z = \begin{pmatrix} 1 & 0 \\ 0 & -1\end{pmatrix}
\end{align*}
\end{definition}

\begin{definition}
\label{bellstates}
Bell States

\begin{align*}
\frac{\ket{00} + \ket{11} }{\sqrt{2}} \\	
\frac{\ket{00} - \ket{11} }{\sqrt{2}} \\	
\frac{\ket{10} + \ket{01} }{\sqrt{2}} \\	
\frac{\ket{01} - \ket{10} }{\sqrt{2}}
\end{align*}
\end{definition}

\begin{definition}
\label{posop}
Positive Operators

Let $A$ be a bounded\footnote{$\Vert Av \Vert \leq M\Vert v \Vert$ for some $M>0$ and all $v \in \mathcal{H}$} linear operator on complex Hilbert space $\mathcal{H}$. The following conditions are equivalent to $A$ being positive

\begin{enumerate}
\item $A=S^\dag S$ for some bounded operator $S$ on $\mathcal{H}$
\item $A$ is hermitian and $\bra{x} A \ket{x} \geq 0$ for every $\ket{x} \in \mathcal{H}$
\item the spectrum of $A$ is non-negative
\end{enumerate}
\end{definition}

\begin{definition}
\label{trop}
Trace of an Operator

Let $\{\ket{i}\}$ be an orthonormal basis for $A$ and so
\begin{align*}
\tr(A) &= \sum_i A_{ii} \\
&= \sum_i \bra{i} A \ket{i}	
\end{align*}

Hence, if we extend $\ket{\psi}$ to the orthonormal basis $\{\ket{i}\}$ which includes $\ket{\psi}$ as the first element (for example via the Gram-Schmidt procedure) then

\begin{align*}
	\tr(A\ket{\psi}\bra{\psi}) &= \sum_i \bra{i} A\ket{\psi}\bra{\psi}\ket{i}	 \\
	&= \bra{\psi} A\ket{\psi}
\end{align*}

by orthonormality.
\end{definition}

\begin{theorem}Spectral Theorem
\label{thm:spec}

Suppose $A$ is a compact\footnote{the image under $A$ acting on any bounded subset of $\mathcal{H}$ is a compact subset of $\mathcal{H}$} hermitian operator (compactness ensures $A$ has eigenvectors) on complex Hilbert space $\mathcal{H}$. Hence, there is an orthonormal basis of $\mathcal{H}$ consisting of eigenvectors of $A$. Each eigenvalue is in $\RR$.	
\end{theorem}
\end{appendices}

\nocite{*}
\printbibliography

\end{document}