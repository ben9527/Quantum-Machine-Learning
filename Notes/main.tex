\documentclass{book}
\usepackage{subfiles}
\usepackage[toc,page]{appendix}

\usepackage{tikz}
\usetikzlibrary{trees}
\tikzset{
  invisible/.style={opacity=0},
  visible on/.style={alt={#1{}{invisible}}},
  alt/.code args={<#1>#2#3}{%
    \alt<#1>{\pgfkeysalso{#2}}{\pgfkeysalso{#3}} % \pgfkeysalso doesn't change the path
  },
  properties/.style={green, ultra thick},
}

\oddsidemargin=17pt \evensidemargin=17pt
\headheight=9pt     \topmargin=26pt
\textheight=564pt   \textwidth=433.8pt
\date{}
\usepackage{url}
\usepackage{amsmath}
\usepackage{amsfonts,amssymb,amsthm,float,graphicx}
\usepackage{physics}
\usepackage{graphicx}
\usepackage{mathtools}
\usepackage{float}
\usepackage{hyperref}
\hypersetup{
    colorlinks=true, %set true if you want colored links
    linktoc=all,     %set to all if you want both sections and subsections linked
    linkcolor=blue,  %choose some color if you want links to stand out
}
\usepackage[backend=biber]{biblatex}
\addbibresource{course_notes.bib}

%new math symbols taking no arguments
\newcommand\0{\mathbf{0}}
\newcommand\CC{\mathbb{C}}
\newcommand\FF{\mathbb{F}}
\newcommand\NN{\mathbb{N}}
\newcommand\QQ{\mathbb{Q}}
\newcommand\RR{\mathbb{R}}
\newcommand\ZZ{\mathbb{Z}}
\newcommand\bb{\mathbf{b}}
\newcommand\kk{\Bbbk}
\newcommand\mm{\mathfrak{m}}
\newcommand\pp{\mathfrak{p}}
\newcommand\xx{\mathbf{x}}
\newcommand\yy{\mathbf{y}}
\newcommand\GL{\mathit{GL}}
\newcommand\into{\hookrightarrow}
\newcommand\nsub{\trianglelefteq}
\newcommand\onto{\twoheadrightarrow}
\newcommand\minus{\smallsetminus}
\newcommand\goesto{\rightsquigarrow}
\newcommand\nsubneq{\vartriangleleft}

%redefined math symbols taking no arguments
\newcommand\<{\langle}
\renewcommand\>{\rangle}
\renewcommand\iff{\Leftrightarrow}
\renewcommand\phi{\varphi}
\renewcommand\implies{\Rightarrow}

%new math symbols taking arguments
\newcommand\ol[1]{{\overline{#1}}}

%redefined math symbols taking arguments
\renewcommand\mod[1]{\ (\mathrm{mod}\ #1)}

%roman font math operators
\DeclareMathOperator\aut{Aut}

%for easy 2 x 2 matrices
\newcommand\twobytwo[1]{\left[\begin{array}{@{}cc@{}}#1\end{array}\right]}

%for easy column vectors of size 2
\newcommand\tworow[1]{\left[\begin{array}{@{}c@{}}#1\end{array}\right]}

\newtheorem{theorem}{Theorem}[section]
\newtheorem{corollary}{Corollary}[theorem]
\newtheorem{lemma}[theorem]{Lemma}
\newtheorem{proposition}[theorem]{Proposition}
\newtheorem{exercise}[theorem]{Exercise}
\newtheorem{definition}[theorem]{Definition}
\newtheorem{fact}[theorem]{Fact}
\newtheorem{algorithm}[theorem]{Algorithm}
\newtheorem{example}[theorem]{Example}

\title{Some results in quantum learning theory}
\author{Faris Sbahi}

\begin{document}
\maketitle

%\chapter*{Abstract}

\tableofcontents

\chapter{Introduction}

In this document, we provide a presentation of the latest results in quantum learning theory alongside theoretical and empirical extensions. 

The first part of our paper is a review: First, we present an overview of learning theory and the mathematical methods necessary for our analyses (whereas the Appendix provides the necessary tools of quantum information theory for the unfamiliar reader). Next, we present a review of the brief history of quantum machine learning. The subsequent part of our paper consists of analyses and extensions of recent results in quantum learning theory: (1) supervised learning using hybrid quantum-classical circuits, (2) Tang's \cite{tang2018quantum} idea of least-square sampling providing parallel classical algorithms for quantum machine learning algorithms that solve singular value transformation problems, and (3) information theoretic bounds on quantum computational learning.

In the first part of our paper, we extend quantum feature maps which seek to solve the quantum encoding problem by encoding data inputs into a quantum state that implicitly performs the feature map given by a kernel function. Therefore, if the kernel is sufficiently difficult to evaluate classically, then there may exist a quantum advantage. Our contribution comes in two parts: (1) we offer a theoretical extension which identifies binary classification with state identification for $m=2$ density matrices. Hence, we propose using the trace distance as a measure to tune hyper-parameters which determine the non-linearity of our feature map, and (2) we evaluate the performance of quantum feature maps using variational circuits on practical datasets, in addition to reproducing the results of \cite{havlicek2018supervised} where they considered kernel-separable data.

In the next part, we apply the methods of Tang to an analysis of a polynomial singular value transformation. In other words, given a low-rank matrix $A \in \RR^{m\times n}$ with SVD $\sum_i \sigma_i \ket{u_i} \bra{v_i}$ and a vector $b \in \RR^n$, assume that we have $\ell^2$ sample and query access to $b$. Then, we can approximately simulate sample and query access to $\sum_i (\sigma_i)^m \ket{u_i} \bra{v_i} b$ for any $m \in \ZZ$.

In the final part, we present the bounds due to \cite{arunachalam2016optimal, arunachalam2018two} by exposing a general method introduced in these works which utilizes principles of quantum information theory to derive sample complexity bounds on various concept classes, oracle access, and learning settings.

%\subfile{crypto_notes.tex}

\subfile{quantum_learning_notes.tex}

\chapter{Acknowledgements} 

First and foremost, a special thank you to my advisor Dr. Marvian for graciously setting aside time to facilitate this journey within quantum information theory. I thank him for the pleasant and insightful conversations, endless inspiration, and impactful lessons on Physics---but also patience, communication, and perspective. I also thank the defense committee members Dr. Sun and Dr. Younes for courteously and enthusiastically participating in the thesis process.

I am grateful to the Department of Physics at Duke University and Dr. Scholberg for facilitating the opportunity to write and present this work in addition to the coursework and instruction that shaped my view of the field. I am appreciative of my academic advisor Dr. Greenside for providing my first introduction to the beauty of Quantum Physics and offering his endless support throughout my Undergraduate career.

Finally and importantly, I thank my family for their never-ending support of my endeavors. 

\appendix
\chapter{Appendix}

\section{Quantum Mechanics}

\begin{definition}
\label{pauli}
Pauli Matrices

\begin{align*}
\sigma_x &= X = \begin{pmatrix} 0 & 1 \\ 1 & 0\end{pmatrix} \\
\sigma_y &= Y = \begin{pmatrix} 0 & -i \\ i & 0\end{pmatrix}\\
\sigma_z &= Z = \begin{pmatrix} 1 & 0 \\ 0 & -1\end{pmatrix}
\end{align*}
\end{definition}

\begin{definition}
\label{bellstates}
Bell States

\begin{align*}
\frac{\ket{00} + \ket{11} }{\sqrt{2}} \\	
\frac{\ket{00} - \ket{11} }{\sqrt{2}} \\	
\frac{\ket{10} + \ket{01} }{\sqrt{2}} \\	
\frac{\ket{01} - \ket{10} }{\sqrt{2}}
\end{align*}
\end{definition}

\begin{definition}
\label{posop}
Positive Operators

Let $A$ be a bounded\footnote{$\Vert Av \Vert \leq M\Vert v \Vert$ for some $M>0$ and all $v \in \mathcal{H}$} linear operator on complex Hilbert space $\mathcal{H}$. The following conditions are equivalent to $A$ being positive

\begin{enumerate}
\item $A=S^\dag S$ for some bounded operator $S$ on $\mathcal{H}$
\item $A$ is hermitian and $\bra{x} A \ket{x} \geq 0$ for every $\ket{x} \in \mathcal{H}$
\item the spectrum of $A$ is non-negative
\end{enumerate}
\end{definition}

\begin{definition}
\label{trop}
Trace of an Operator

Let $\{\ket{i}\}$ be an orthonormal basis for $A$ and so
\begin{align*}
\tr(A) &= \sum_i A_{ii} \\
&= \sum_i \bra{i} A \ket{i}	
\end{align*}

Hence, if we extend $\ket{\psi}$ to the orthonormal basis $\{\ket{i}\}$ which includes $\ket{\psi}$ as the first element (for example via the Gram-Schmidt procedure) then

\begin{align*}
	\tr(A\ket{\psi}\bra{\psi}) &= \sum_i \bra{i} A\ket{\psi}\bra{\psi}\ket{i}	 \\
	&= \bra{\psi} A\ket{\psi}
\end{align*}

by orthonormality.
\end{definition}

\begin{theorem}Spectral Theorem
\label{thm:spec}

Suppose $A$ is a compact\footnote{the image under $A$ acting on any bounded subset of $\mathcal{H}$ is a compact subset of $\mathcal{H}$} hermitian operator (compactness ensures $A$ has eigenvectors) on complex Hilbert space $\mathcal{H}$. Hence, there is an orthonormal basis of $\mathcal{H}$ consisting of eigenvectors of $A$. Each eigenvalue is in $\RR$.	
\end{theorem}

\subfile{nielsen_chuang_notes.tex}

\nocite{*}
\printbibliography

\end{document}