\documentclass[main.tex]{subfiles}

\begin{document}
\section{Algorithms for solving linear systems of equations}

One such application of Phase Estimation (Section \ref{phase_estimation}) is with respect to solving linear systems of equations. This is the so-called HHL algorithm \cite{lloyd2010quantum}.

The general problem statement of a linear system is if we are given matrix $A$ and unit vector $\vec{b}$, then find $\vec{x}$ satisfying, $A\vec{x} = \vec{b}$. 

However, assume that instead of solving for $x$ itself, we instead solve for an expectation value $x^T M x$ for some linear operator $M$. Hence, one can show that our algorithm has a runtime bound of $O(\log(N)\kappa ^{2})$, if we can further assume that the linear system is sparse and has a low condition number $\kappa$.

So, assume that $A$ in our linear system is an $N \times N$ Hermitian matrix. Notice that this is an "unrestrictive" constraint on $A$ because we can always take non-Hermitian matrix $A'$ and linear system $A' \vec{x} = \vec{b}$ and instead solve $\begin{bmatrix}
	0 && A' \\ A'^\dag && 0
\end{bmatrix} \begin{bmatrix} 0 \\ x \end{bmatrix} = \begin{bmatrix} b \\ 0 \end{bmatrix}$. Hence, we we will assume that $A$ is Hermitian from here on. 

Recall that because $A$ is hermitian $\implies$ we can perform quantum phase estimation using $e^{-iAt}$ as the unitary transformation. This can be done efficiently if $A$ is sparse.

So, we first prepare $\ket{b}$ (the representation of $\vec{b}$). We assume that this can be done efficiently or that $\ket{b}$ is supplied as an input.

Denote by $\ket{\psi_j}$ the eigenvectors of $A$ with associated eigenvalues $\lambda_j$. Hence, we can express $\ket{b}$ as $\ket{b} = \sum_j \beta_j \ket{\psi_j}$.  So, we initialize a first register to state $\sum_j \beta_j \ket{\psi_j}$ and second register to state $\ket{0}$ . After applying phase estimation, we then have the joint state $\sum_j \beta_j \ket{\psi_j} \ket{\widetilde{\lambda}_j}$, where $\widetilde{\lambda}_j$ is an approximation of $\lambda_j$. We'll assume that this approximation is perfect from here on. 

Next we add an ancilla qubit and perform a rotation conditional on the first register while now holds $\ket{\lambda_j}$. The rotation transforms the system to

\begin{align*}
\sum_j \beta_j \ket{\psi_j} \ket{\lambda_j} \Big(\sqrt{1-\frac{C^2}{\lambda_j^2}}\ket{0} + \frac{C}{\lambda_j}\ket{1}\Big)
\end{align*}

for some small constant $C \in \RR$ that is $O(1/\kappa)$.

Hence, we can undo phase estimation to restore the second register to $\ket{0}$.

Now, if we measure the ancillary qubit in the computational basis, we'll evidently collapse the state to $\ket{1}$ with some probability. We'd then have

\begin{align*}
	\sum_j \frac{C}{\lambda_j} \beta_j \ket{\psi_j} \ket{\lambda_j}\ket{1} = C (A^{-1} \ket{b})
\end{align*}

In particular, the probability of getting this result is 

\begin{align*} 
	p(-1) &= \Bigg(\sum_j \beta_j \bra{\psi_j} \bra{\lambda_j} \Big(\sqrt{1-\frac{C^2}{\lambda_j^2}}\bra{0} + \frac{C}{\lambda_j}\bra{1}\Big) \Bigg)\ket{1}\bra{1} \Bigg(\sum_j \beta_j \ket{\psi_j} \ket{\lambda_j} \Big(\sqrt{1-\frac{C^2}{\lambda_j^2}}\ket{0} + \frac{C}{\lambda_j}\ket{1}\Big)\Bigg) \\
	&= \sum_j \beta_j \bra{\psi_j} \bra{\lambda_j} \Big(\sqrt{1-\frac{C^2}{\lambda_j^2}}\bra{0} + \frac{C}{\lambda_j}\bra{1}\Big) \ket{1}\bra{1} \beta_j \ket{\psi_j} \ket{\lambda_j} \Big(\sqrt{1-\frac{C^2}{\lambda_j^2}}\ket{0} + \frac{C}{\lambda_j}\ket{1}\Big) \\
	&= \sum_j \beta_j \bra{\psi_j} \bra{\lambda_j} \frac{C}{\lambda_j}\bra{1}\ket{1}\bra{1} \beta_j \ket{\psi_j} \ket{\lambda_j} \frac{C}{\lambda_j}\ket{1} \\
	&= \sum_j \beta_j^2 \frac{C^2}{\lambda_j^2} \\
	&= \| A^{-1} \ket{b} \|^2 C^2 = O(1/\kappa^4)
\end{align*} 

Finally, we can make a measurement $M$ whose expectation value $\bra{x}M\ket{x}$ corresponds to the feature of $x$ we wish to evaluate. 

\section{Supervised learning with quantum enhanced feature spaces}

\subsection{Prelude}

We are given data from a training set $T$ and a test set $S$ of a subset $\Omega \subset \RR^d$. We assume that $S$ and $T$ are drawn from the same input space $X$. Furthermore, there exists output space $Y = \{ -1, +1 \} $ and a distribution $D$ on $X \times Y$.

Now, suppose we have a labelling $m: T \cup S \rightarrow Y$.  Our goal is to use this information to find some approximation function $\tilde{f} : X \rightarrow Y$ that minimizes estimation error for function class $F$. In other words, let true risk for function $f$ be defined as

\begin{align*}
R^{true}(f) = P_{X, Y \sim D}(f(X) \neq Y)	
\end{align*}

Then, estimation error is the difference in true risk between $\tilde{f}$ and optimal choice $f^* = \inf_{f \in F}R^{true}(f)$.

One classical method is using so-called Support Vector Machines (SVM), which construct a separating hyperplane such that the distance to the nearest training observation (minimum margin) is maximized. Much of the popularity of SVMs can be attributed to its association with the "kernel trick" which maps the data to a higher dimensional space so that it is separable or approximately separable.

Here, we suppose that the data is given classically and we seek to show that, in some cases, we can obtain a quantum advantage by either generating the separating hyperplane in quantum feature space or simply estimating the kernel function.

\subsection{Feature Map}

Consider the feature vector kernel $K(x, z) = | \bra{\Phi(x)}\ket{\Phi(z)} |^2$

%\section{Density Matrix Exponentiation Algorithms}
%\url{https://www.nature.com/articles/nphys3029}
%
%\section{Review of Quantum Machine Learning}
%\url{https://www.nature.com/articles/nature23474}
%
%\section{Learnability of Quantum States}
%\begin{enumerate}
%\item \url{https://arxiv.org/abs/quant-ph/0608142}
%\item \url{https://arxiv.org/abs/1711.01053}
%\item \url{https://arxiv.org/abs/1801.05721}
%\end{enumerate}

\section{Singular Value Transformation using Length-Square Sampling Methods}

\subsection{Stochastic Regression}

\subsubsection{Definitions and Assumptions}

Let $b \in \CC^m$ and $A \in \CC^{m \times n}$ s.t. $\Vert A \Vert \leq 1$ where $\Vert \cdot \Vert$ signifies the operator norm (or spectral norm). Furthermore, require that $\rank(A) = k$ and $\Vert A^+ \Vert \leq \kappa$ where $A^+$ is the pseudoinverse of $A$. Hence, observe that $\Vert A \Vert \leq 1$ is equivalent to $A$ having maximum singular value $1$\footnote{To see this, simply consider Spectral Theorem applied to Hermitian matrix $A^\dag A$}. Similarly, $A^+$ has inverted singular values from $A$ and so $\Vert A^+ \Vert$ is equal to the reciprocal of the minimum nonzero singular value. Therefore, the condition number of $A$ is given by $\Vert A \Vert \Vert A^+ \Vert \leq \kappa$.

So, define $x$ to be the least-squares solution to the linear system $Ax = b$ i.e. $x = A^+ b$. Then, in terms of these definitions, we define two primary goals:

\begin{enumerate}
\item Query a vector $\tilde{x}$ s.t. $\Vert \tilde{x} - x \Vert \leq \epsilon \Vert x \Vert$
\item Sample from a distribution that approximates $\frac{|x_j|^2}{\Vert x \Vert^2}$ within total variation distance (\autoref{def:tve}) $2\epsilon$.
\end{enumerate}

In order to do this, we simply assume that we have length-square sampling access to $A$. In other words, we are able to sample row indices of $A$ from the distribution $\frac{\Vert A_{(i, \cdot)}\Vert^2}{\Vert A \Vert^2_F}$

\subsubsection{Sequence of Approximations}

First, we'll summarize the sequence of approximations that we'll perform using length-squared sampling techniques. We'll describe these steps in depth in the following sections.

Of course, we know that the least squares solution of the linear system is given by the orthogonal projection

\begin{align*}
	(A^\dag A)^+ A^\dag = A^+ b
\end{align*}

So, we first approximate $A^\dag A$ by $R^\dag R$ where $R \in \CC^{r \times n}$, $r \ll m$ is constructed from length-square sampling $r$ rows of $A$. Now, denote the spectral decomposition 

\begin{align*}
A^\dag A \approx R^\dag R = \sum_{l=1}^k \frac{1}{\sigma_l^2}\ket{v^{(l)}}\bra{v^{(l)}}
\end{align*}

where of course $\sigma_i$ and $\ket{v^{(i)}} \in \CC^n$ are the singular values and right singular vectors of $R$, respectively.

We see that computing these right singular vectors of $R$ can still be computationally prohibitive given the dimension $n$. Hence, we can use length-square sampling again, this time on the columns of $R$ to give a matrix $C \in \CC^{r \times c}$, $c \ll n$. Now, the left singular vectors of $C$ which we denote as $\ket{w^{(i)}} \in \CC^r$ can be efficiently computed via standard SVD methods. So,

\begin{align*}
RR^\dag \approx CC^\dag = \sum_{l=1}^k \frac{1}{\sigma_l^2}\ket{w^{(l)}}\bra{w^{(l)}}
\end{align*}


We can then show that ()

\begin{align}
\label{def:approx-right}
\ket{\tilde{v}^{(i)}} := R^\dag \ket{w^{(l)}} / \tilde{\sigma}_l
\end{align}

provides a good approximation of $\ket{v^{(i)}}$. Note that $\tilde{\sigma}_l$ are the singular values of $C$ which then approximate the singular values of $R$ which similarly approximate the singular values of $A$. This follows from $A^\dag A \approx R^\dag R$ and $RR^\dag \approx CC^\dag$ by the Hoffman--Wielandt inequality detailed in Lemma 2.7 of \cite{kannan2017randomized} and stated without proof below.

\begin{lemma}Hoffman--Wielandt inequality

	If $P, Q$ are two real, symmetric $n \times n$ matrices and $\lambda_1, \cdots \lambda_n$ denote eigenvalues in non-decreasing order, then
	
	\begin{align*}
		\sum_{t=1}^n(\lambda_t(P) - \lambda_t(Q))^2 \leq \Vert P - Q \Vert_F^2
	\end{align*}
\end{lemma}


At this point, it seems like we haven't made much progress since computing $R^\dag \ket{w^{(l)}}$ is still expensive. However, it turns out that all we need to enable query access to $\tilde{x}$ is the ability to efficiently estimate the trace inner product $\tr(U^\dag V)$ where $U$ and $V$ are operators such that $U$ can be the length-square sampled and $V$ can be queried. To see this, we write our solution, $\tilde{x}$, in terms of the approximations thus far

\begin{align*}
	\tilde{x} &\approx A^+ \ket{b} \\
	&\approx (R^\dag R)^+ A^\dag \ket{b}\\
	&\approx \sum_{l = 1}^k \frac{1}{\tilde{\sigma}_l^2} \ket{\tilde{v}^{(l)}} \bra{\tilde{v}^{(l)}} A^\dag \ket{b}
\end{align*}

Hence, define $U := A$, $V := \ket{b}\bra{\tilde{v}^{(l)}}$ in which case 

\begin{align*}
\tr(U^\dag V) &= \tr(A^\dag \ket{b} \bra{\tilde{v}^{(l)}}) \\
&= \tr(\bra{\tilde{v}^{(l)}} A^\dag \ket{b} )\\
&= \bra{\tilde{v}^{(l)}} A^\dag \ket{b}
\end{align*}

since $\bra{\tilde{v}^{(l)}} A^\dag \ket{b}$ is a scalar. Therefore, say that 

$$
\tilde{\lambda}_l \approx \tr(A^\dag \ket{b} \bra{\tilde{v}^{(l)}})
$$

and assume that we can compute and memoize these scalars $\tilde{\lambda}_i$ efficiently. In which case,

\begin{align*}
\tilde{x} &\approx \sum_{l = 1}^k \frac{1}{\tilde{\sigma}_l^2} \ket{\tilde{v}^{(l)}} \tilde{\lambda}_l
\intertext{Recalling the definition of $\ket{\tilde{v}^{(i)}}$ (\ref{def:approx-right}),}
&= \sum_{l = 1}^k \frac{1}{\tilde{\sigma}_l^3} R^\dag \ket{w^{(l)}} \tilde{\lambda}_l\\
&= R^\dag \sum_{l = 1}^k \frac{1}{\tilde{\sigma}_l^3} \ket{w^{(l)}} \tilde{\lambda}_l
\intertext{and so defining $z := \sum_{l = 1}^k \frac{1}{\tilde{\sigma}_l^3} \ket{w^{(l)}} \tilde{\lambda}_l$,}
&= R^\dag z
\end{align*}

We see that we can compute $z$ efficiently (and memoize it for future queries) because it is a $k$-linear combination of left singular vectors in $\CC^r$. So, say that we wish to query an element $\tilde{x}_j$. We can simply query column $R_{\cdot, j} \in \CC^r$ (or equivalently row $R_{j, \cdot}^\dag$) and compute $R_{\cdot, j} \cdot z$. Hence, we've achieved our first goal.

In order to achieve our second goal, enabling sample access to a distribution that approximates $\frac{|x_j|^2}{\Vert x \Vert^2}$, we require one more trick: rejection sampling which we detail in Section ().

All in all, we've performed the chain of approximations,

\begin{align*}
	\ket{x} &= A^+ \ket{b} = (A^\dag A)^+ A^\dag \ket{b}\\
	&\approx (R^\dag R)^+ A^\dag \ket{b} = \sum_{l = 1}^k \frac{1}{\tilde{\sigma}_l^2} \ket{v^{(l)}} \bra{v^{(l)}} A^\dag \ket{b}\\
	&\approx \sum_{l = 1}^k \frac{1}{\tilde{\sigma}_l^2} \ket{\tilde{v}^{(l)}} \bra{\tilde{v}^{(l)}} A^\dag \ket{b}\\ %\qquad \Big(\ket{\tilde{v}^{(l)}} := R^\dag \ket{w^{(l)}}\Big)
	&\approx \sum_{l = 1}^k \frac{1}{\tilde{\sigma}_l^2} \ket{\tilde{v}^{(l)}} \tilde{\lambda}_l = R^\dag \sum_{l = 1}^k \frac{1}{\tilde{\sigma}_l^3} \ket{w^{(l)}} \tilde{\lambda}_l = R^\dag z
\end{align*}


Now that we've sketched the steps of this process, we detail each approximation and show that we can achieve the claimed correctness and complexity bounds.

\subsubsection{Computing Approximate Singular Vectors}

As described above, we begin by length-square sampling the original matrix $A \in \CC^{m \times n }$. Suppose we want to draw $s$ rows in $s$ i.i.d. trials. Then, pick row index $i$ of $A$ with probability

\begin{align}
\label{def:A-prob}
p_i = \frac{\Vert A_{(i, \cdot)}\Vert^2}{\Vert A \Vert_F^2} 	
\end{align}

and output random row 

\begin{align*}
 	Y &= \frac{1}{\sqrt{s p_i}} \bra{A_{(i, \cdot)}} \\\
 	&= \frac{1}{\sqrt{s}}\frac{\Vert A\Vert_F}{\Vert A_{(i, \cdot)}\Vert}\bra{A_{(i, \cdot)}}
\end{align*}

which is just a scaling of the $i$th row of $A$\footnote{The reason that we scale by $s$ is so that the expectations of $A^\dag A$ and $R^\dag R$ coincide in the theorem that follows. The reason that we scale by $p_i$ is so that the norms of all rows are equivalent---a fact which we'll utilize when we sample $R$ column-wise.}. In other words,

\begin{align*}
\Pr(Y = \frac{1}{\sqrt{s p_i}} \bra{A_{(i, \cdot)}}) = p_i
\end{align*}


After sampling $s$ rows, we implicitly define matrix $R$ to be the concatenation of the outputted random rows. Therefore,

\begin{align}
\label{def:R}
R &= \begin{bmatrix}
Y_1 \\
Y_2 \\
\vdots \\
Y_s
\end{bmatrix} \in \CC^{s \times n}
\end{align}

Note that $\bra{Y_k}$ denotes the random row outputted by the procedure on the $k$th i.i.d. draw.

\begin{lemma}
	Let $X = Y^\dag Y - E[Y^\dag Y]$ which evidently satisfies $E[X] = 0$. Then,
	
	\begin{align}
	E[X^2] &\preceq E[(Y^\dag Y)^2] \\
	&= A^\dag A \Vert A \Vert_F^2 \frac{1}{s^2}	
	\end{align}
	
	and so 
	
	\begin{align}
	\label{eq:expect-x-2}
	\Vert E[X^2] \Vert_2 \leq \frac{1}{s^2} \Vert A \Vert_2^2 \Vert A \Vert_F^2	
	\end{align}

	
	Furthermore,
	
	\begin{align}
	\Vert X \Vert_2 = \frac{1}{s} \Vert A \Vert^2_F	
	\end{align}
\begin{proof}
First, observe that $E[X^2]$ is the element-wise variance of $Y^\dag Y$ and $E[(Y^\dag Y)^2]$ is the corresponding second moment. Hence, the relation $E[X^2] \leq E[(Y^\dag Y)^2]$ holds element-wise which implies that the matrix relation $E[X^2] \preceq E[(Y^\dag Y)^2]$ holds as well.

Furthermore, 

\begin{align*}
E[(Y^\dag Y)^2] &= \frac{1}{s^2} \sum_{i = 1}^m \frac{p_{i}}{p_{i}^2} \ket{A_{(i, \cdot)}}\bra{A_{(i, \cdot)}}\ket{A_{(i, \cdot)}}\bra{A_{(i, \cdot)}} \\
&= \frac{1}{s^2} \sum_{i = 1}^m \frac{\Vert A \Vert_F^2}{\bra{A_{(i, \cdot)}}\ket{A_{(i, \cdot)}}} \ket{A_{(i, \cdot)}}\bra{A_{(i, \cdot)}}\ket{A_{(i, \cdot)}}\bra{A_{(i, \cdot)}}\tag{using (\ref{def:A-prob})}\\
&= \frac{1}{s^2} \sum_{i = 1}^m \Vert A \Vert_F^2 \ket{A_{(i, \cdot)}}\bra{A_{(i, \cdot)}} = \frac{\Vert A \Vert_F^2}{s^2} \sum_{i = 1}^m  \ket{A_{(i, \cdot)}}\bra{A_{(i, \cdot)}} \\
&= A^\dag A \Vert A \Vert_F^2 \frac{1}{s^2}	
\end{align*}

Recall that $E[R^\dag R] = \frac{1}{s} A^\dag A$. So, observe that

\begin{align*}
\Vert X \Vert_2 &= \Vert Y^\dag Y - E[Y^\dag Y] \Vert_2 \\
&= \frac{1}{s}\Bigg\Vert\frac{1}{p_i}\ket{A_{(i, \cdot)}}\bra{A_{(i, \cdot)}} - A^\dag A \Bigg\Vert_2 \\
&\leq \frac{1}{s} \max\Big\{ \Big\Vert\frac{1}{p_i}\ket{A_{(i, \cdot)}}\bra{A_{(i, \cdot)}}\Big\Vert_2 ,  \Big\Vert A^\dag A \Big\Vert_2\Big\}\\
&\leq \frac{1}{s} \max\Big\{ \frac{\Vert A \Vert_F^2}{\Vert \ket{A_{(i, \cdot)}} \Vert_2^2}\Big \Vert \ket{A_{(i, \cdot)}}\bra{A_{(i, \cdot)}}\Big\Vert_2 ,  \Vert A^\dag A \Vert_F^2\Big\} \intertext{using $\Vert A^\dag A \Vert_2 = \Vert A \Vert_2^2 \leq \Vert A \Vert_F^2$ and plugging in (\ref{def:A-prob}).}\\
&= \frac{1}{s} \Vert A \Vert^2_F	
\end{align*}
\end{proof}
\end{lemma}


\begin{proposition}
\label{prop:expect-exp-mat-norm}
	If $t > 0, t \in \CC$ satisfies $\Vert tX \Vert_2 \leq 1$ for all possible values of $X$, then 
	
	\begin{align}
	\begin{split}
	\Vert E[e^{\pm tX}] \Vert_2 &\leq 1 + \frac{t^2}{s^2} \Vert A \Vert_2^2 \Vert A \Vert_F^2 \\
	&\leq e^{t^2 \Vert A\Vert_2^2 \Vert A \Vert_F^2 / s^2}	
	\end{split}
	\end{align}
	
	\begin{proof}
		First, from (\ref{lemma:exp-eigen-approx}) we know that $E[e^{tX}] \preceq E[I + X + X^2] = I + E[X^2]$ since $E[X] = 0$. Hence, we have the proposition by (\ref{eq:expect-x-2}).
	\end{proof}
\end{proposition}

\begin{theorem}
Let $A \in \CC^{m \times n}$ and $R \in \CC^{r \times n}$ be constructed by the length-square sampling and scaling so that $E[R^\dag R] = E[A^\dag A]$ (requirements that are met by $R$ defined in (\ref{def:R})). Then, for all $\epsilon \in [0, \Vert A \Vert / \Vert A \Vert_F]$\footnote{If $\epsilon \geq \Vert A \Vert / \Vert A \Vert_F$, then we can simply use $\hat{0}$ to approximate $A^\dag A$}, we have

\begin{align*}
\Pr(\Vert R^\dag R - A^\dag A \Vert \geq \Vert A \Vert \Vert A \Vert_F) \leq 2ne^{\frac{- \epsilon^2 s}{4}}	
\end{align*}

Hence, for $s \geq (4 \ln \frac{2n}{\eta}) / \epsilon^2$, with probability at least $(1 - \eta)$ we have

\begin{align*}
	\Vert R^\dag R - A^\dag A \Vert \leq \epsilon\Vert A \Vert \Vert A \Vert_F
\end{align*}

\begin{proof}
	From our definition above, we have that 
	
	\begin{align*}
		R^\dag R &= \begin{bmatrix}
Y_1^\dag &
Y_2^\dag &
\hdots &
Y_s^\dag
\end{bmatrix}\begin{bmatrix}
Y_1 \\
Y_2 \\
\vdots \\
Y_s
\end{bmatrix} \\
&= \sum_{k=1}^s \ket{Y_k}\bra{Y_k} \intertext{Let $i_k$ give the index of the row sampled from $A$ on the $k$th draw. Hence, }\\
&= \frac{1}{s} \sum_{k=1}^s \frac{1}{p_{i_k}} \ket{A_{(i_k, \cdot)}}\bra{A_{(i_k, \cdot)}}
\end{align*}

Furthermore,

\begin{align*}
E[R^\dag R] &= 	\frac{1}{s} \sum_{k=1}^s \sum_{i_k = 1}^m \frac{p_{i_k}}{p_{i_k}} \ket{A_{(i_k, \cdot)}}\bra{A_{(i_k, \cdot)}} \\
&= \frac{1}{s} \sum_{k=1}^s A^\dag A \\
&= A^\dag A
\end{align*}

Note that similarly $E[Y^\dag Y] = \frac{1}{s} A^\dag A$.

So, we can define $X_i = \ket{Y_k}\bra{Y_k} - E\Big[\ket{Y_k}\bra{Y_k}\Big]$ which is evidently an i.i.d. copy of $X$ as we've defined previously. Hence,

\begin{align*}
\sum_{i=1}^s X_i &= R^\dag R - E[R^\dag R] \\
&= R^\dag R - A^\dag A	
\end{align*}

Now, we can first apply Theorem \ref{thm:chern-mat-eigen} with $a = \epsilon \Vert A \Vert_2 \Vert A \Vert_F$,

\begin{align*}
\Pr(\Big\Vert \Big(\sum_{i=1}^s X_i\Big)\Big\Vert_2 \geq \epsilon \Vert A \Vert_2 \Vert A \Vert_F ) \\ &\leq ne^{-t \epsilon \Vert A \Vert_2 \Vert A \Vert_F} (\Vert E[e^{tX}]\Vert_2^s + \Vert E[e^{-tX}]\Vert_2^s)	
\intertext{for any $t > 0$. Hence, we can apply Proposition \ref{prop:expect-exp-mat-norm} which then gives us}\\
	&\leq 2ne^{-t \epsilon \Vert A \Vert_2 \Vert A \Vert_F} e^{t^2 \Vert A\Vert_2^2 \Vert A \Vert_F^2 / s^2}	
	\intertext{for $t \leq s / \Vert A \Vert_F^2$. Hence, we can set $t = \frac{\epsilon s}{2 \Vert A \Vert_F \Vert A \Vert_2}$ (which is indeed less than $s / \Vert A \Vert_F^2$) and finally,}
	&\leq 2ne^{- \epsilon^2 s / 4}
\end{align*}

Therefore, if we require that $s \geq (4 \ln \frac{2n}{\eta}) / \epsilon^2$ we then have

\begin{align*}
\Pr(\Big\Vert \Big(\sum_{i=1}^s X_i\Big)\Big\Vert_2 \geq \epsilon \Vert A \Vert_2 \Vert A \Vert_F ) &\leq 2ne^{- \frac{\epsilon^2}{4} \frac{4 \ln \frac{2n}{\eta}}{ \epsilon^2}} \\
&= 2ne^{- \ln \frac{2n}{\eta}} = \eta
\end{align*}
\end{proof}
\end{theorem}

\section{Optimal Quantum Sample Complexity}

This paper\cite{arunachalam2016optimal} provides an instructive example of how one can use quantum information theory to discuss the ability of a quantum learning algorithm to learn from a distribution of quantum states. Perhaps, this is unsurprising given the surface-level connection with quantum state discrimination which has been closely studied in quantum information theory.

\subsection{Definitions}

\subsubsection{Quantum Learning Models: PAC Setting}

a quantum example oracle $QPEX(c,D)$ acts on $\ket{0}^{\otimes n}\ket{0}$ and produces a quantum example $\sum_{x\in\{0,1\}^n} D(x)\ket{x,c(x)}$.

A quantum learner is given access to some copies of the state generated by $QPEX(c,D)$ and performs a POVM where each outcome is associated with a hypothesis. 

A learning algorithm $A$ is an $(\epsilon, \delta)$-PAC quantum learner for $C$ if for every $c \in C$ and distribution $D$, given access to the $QPEX(c,D)$ oracle, $A$ outputs an $h$ such that $err_D (h, c) \leq \epsilon$, with probability at least $1 - \delta$.

The sample complexity of $A$ is the maximum number invocations of the $QPEX(c,D)$ oracle, maximized over all $c \in C$ , distributions $D$, and the learner’s internal randomness. The $(\epsilon, \delta)$-PAC quantum sample complexity of a concept class $C$ is the minimum sample complexity over all $(\epsilon,\delta)$-PAC quantum learners for $C$.

\subsubsection{Quantum Learning Models: Agnostic Setting}

For a joint distribution $D : \{0, 1\}^{n+1} \rightarrow [0, 1]$ over the set of examples, the learner has access to an $QAEX(D)$ oracle which acts on $\ket{0}^{\otimes n}\ket{0}$ and produces a quantum example $\sum_{(x, b)\in\{0,1\}^{n+1}} D(x,b)\ket{x,b}$. 

A learning algorithm A is an $(\epsilon, \delta)$-agnostic quantum learner for $C$ if for every distribution $D$, given access to the $QAEX(D)$ oracle, $A$ outputs an $h \in C$ such that $err_D (h) \leq opt_D(h) + \epsilon$ with probability at least $1 - \delta$.

The sample complexity of $A$ is the maximum number invocations of the $QAEX(D)$ oracle over all distributions $D$ and over the learner’s internal randomness. The $(\epsilon, \delta)$-agnostic quantum sample complexity of a concept class $C$ is the minimum sample complexity over all $(\epsilon, \delta)$-agnostic quantum learners for $C$.


\begin{subappendices}
\subsection{Pretty Good Measurement (PGM)}

Given a set of density matrices $\{\sigma_i\}$ and a quantum state $\rho$ we are promised that $\rho$ is in state $\sigma_i$ with probability $p_i$. In the general case we have $i \in [m]$ and of course $\sum_{i=1}^m p_i = 1$. Our goal is then to successfully identify which of the $\sigma_i$ that our state $\rho$ is actually in. This is known as Quantum Hypothesis Testing. 

In some sense, thinking back to Holevo's Theorem (Theorem \ref{thm:holevo}), this is related to Bob attempting to access information transported from Alice that is given from the distribution above.

Hence, we perform a maximization with respect to both the probabilities on each state as well as with respect to any randomness that our approach employs. We then must choose a Quantum POVM $\{E_i\}$ that carries put a measurement and maximizes our probability of getting the state right.

So say we pick a POVM. Hence, we know that

\begin{align*}
\Pr(Success)= \sum_i p_i \Tr (\sigma_i E_i)
\end{align*}


So this is the quantity that we seek to maximize. 

We've shown above that the trace distance provides the solution for $m=2$. As it turns out for $m>2$ this is not an easy problem. However, PGM provides a sound approximation:

Intuitively, it might seem reasonable to simply choose 

$$E_i = p_i \sigma_i$$

Unfortunately, then, $\sum_i E_i \neq I$. Well, one case we may think about is if we can guarantee $\sigma_i$ is a the sum of pure states from an orthonormal basis. In which case, let $S = \sum_i \sigma_i$ and we choose 

$$E_i = S^{1/2} \sigma_i S^{1/2} $$ 

by orthonormality. Then we have

$$
\tr(E_i \sigma_i) = \tr(\sigma_i) = 1
$$

Inspired by this, define the PGM POVM to be $E_i = S^{-1/2}p_i \sigma_i S^{-1/2}$ for our original problem.

\subsection{Trace Distance}

\begin{definition}
Trace Distance

The trace distance $T(\cdot , \cdot)$ is a metric on the space of density operators and gives a measure of distinguishability between states. In particular, let $\rho, \sigma$ be density operators,

\begin{align*}
	T(\rho, \sigma) &= \frac{1}{2} \Tr [\sqrt{(\rho - \sigma)^2}] \\
	&= \frac{1}{2} \sum_i |\lambda_i|
\end{align*}

where $\lambda_i$ are the eigenvalues of Hermitian $\rho - \sigma$.

\end{definition}

\begin{lemma}
For any states $\rho, \sigma$ one may write $\rho - \sigma = Q - S$ where $Q$ and $S$ are positive operators with support on orthogonal vector spaces	(Exercise 9.7 \cite{nielsen2010quantum})
\end{lemma}

\begin{proof}
$\rho, \sigma$ are p.s.d operators. Hence, $\rho - \sigma$ is Hermitian, so we can write $\rho - \sigma = \sum_i \lambda_i \ket{u_i}\bra{u_i}$ where $\{ u_i \}$ is an orthonormal basis of the Hilbert space, by spectral theorem. Now, we can decompose the eigenbasis into positive and negative components. Then, $\rho - \sigma = \sum_i \lambda^+_i \ket{u_i}\bra{u_i} + \sum_j \lambda^-_j\ket{u_j}\bra{u_j}$ where $\lambda^+_i >0, \lambda^-_j < 0$. Since, each component partitions the vector space (other than at the additive identity) by the orthogonality condition, this is a direct sum.
\end{proof}


\begin{lemma}
The maximum probability of distinguishing between two states with an optimal measurement is given by

\begin{align*}
	1/2[1 + T(\rho_1, \rho_2)]
\end{align*}
	
\end{lemma}
\begin{proof}
Say that we have the ensemble $\{ (p_1, \rho_1), (p_2, \rho_2)\}$. We seek to define a POVM $\{E_1, E_2 \}$ where $E_1$ indicates $\rho_1$ and similarly for $\rho_2$. hence the probability of success is given by

\begin{align*}
\Pr_{\max} &= \max_{E_1, E_2}p_1\Tr[E_1\rho_1] + p_2\Tr[E_2\rho_2] \\
&= \max_{E_1} p_1\Tr[E_1\rho_1] + p_2\Tr[(I - E_1)\rho_2] \tag{completeness of POVM} \\
&= \max_{E_1}p_2\Tr[\rho_2] + \Tr[E_1(p_1\rho_1 - p_2\rho_2)] \tag{linearity of trace} \\
&= \max_{E_1} \Big( p_2 + \Tr[E_1(p_1\rho_1 - p_2\rho_2)]\Big) \tag{$\Tr(\rho) = 1$ for any density operator}
\end{align*}

Therefore, the optimal projection $E_1$ is onto the positive eigenspace of $(p_1\rho_1 - p_2\rho_2)$. In which case, recalling that $\rho_i$ are p.s.d, $E_1(p_1\rho_1 - p_2\rho_2)] = p_1 \rho_1$ and so

\begin{align*}
&= 	\Big( p_2 + \Tr[p_1\rho_1]\Big)
\end{align*}
 
	
We did implicitly assume that $E_1 + E_2 = I$ where we could've had an additional indeterminate $E_3$. However, the proof would still follow in any case (\cite{nielsen2010quantum}).
\end{proof}


\begin{lemma}
\label{lem:psd-trace}
Let $A, B, C$ by symmetric $d \times d$ matrices satisfying $A \succeq 0$ and $B \preceq C$. Hence, $\Tr(AB) \leq \Tr(AC)$

\begin{proof}
	Write $A$ in its spectral decomposition $A = \sum \lambda_i \ket{i}\bra{i}$, invoking Spectral Theorem (\ref{thm:spec}). Hence,
	
	\begin{align*}
		\Tr(AB) &= \Tr(\sum \lambda_i \ket{i}\bra{i} B)\\
		&= \sum \lambda_i \Tr(\ket{i}\bra{i} B) \tag{linearity of trace}\\
		&= \sum \lambda_i \Tr(\bra{i} B \ket{i}) \tag{cyclic property of trace}\\
		&\leq \sum \lambda_i \Tr(\bra{i} C \ket{i})\\
		&= \sum \lambda_i \Tr(\ket{i}\bra{i} C) = \Tr(\sum \lambda_i \ket{i}\bra{i} C) = \Tr(AC)
	\end{align*}
\end{proof}
\end{lemma}

\begin{corollary}
\label{cor:psd-tr-norm-ineq}
If $A, B \succeq 0$, then $\Tr(AB) \leq \Vert B \Vert_2 \Tr(A)$

\begin{proof}
Note that the singular values of $B$ coincide with the eigenvalues of $B$ since $B^\dag B = B^2$ and $B \succeq 0$ $\implies$ $\lambda_i(B) \geq 0$, $\forall i$. So, let $C = \Vert B \Vert_2 I$ which then trivially satisfies $\lambda_i(C) = \lambda_{\max}(B)$, $\forall i$ since $C$ is the diagonal matrix with diagonal values all equal to $\lambda_{\max}(B)$. Therefore, $B \preceq C$. So, we can simply apply \ref{lem:psd-trace} above,

\begin{align*}
\Tr(AB) &\leq \Tr(AC) \\
&= \Tr\big(A \Vert B \Vert_2 I\big)\\
&= \Vert B \Vert_2 \Tr(A)
\end{align*}
\end{proof}
\end{corollary}

\begin{definition} Total Variation Distance.
\label{def:tve}

Let $P$ and $Q$ be distinct probability measures on a $\sigma$-algebra $\mathcal{F}$ of subsets of the sample space $\Omega$. Then, the total variation distance is given by

\begin{align*}
\delta(P, Q) &= \sup_{A \in \mathcal{F}}\vert P(A) - Q(A)\vert
\end{align*}
\end{definition}

\begin{lemma}Hoeffding--Chernoff Inequality
\label{lem:chernoff}

Let $X_1, X_2, \cdots, X_s$	be i.i.d real random variables. For any positive, real numbers $a, t$ we have that, from Markov's inequality,

\begin{align*}
\Pr(\sum_{i=1}^s X_i \geq a) &\leq e^{-ta} E\Bigg[\prod_{i=1}^s e^{tX_i}\Bigg]\\
&= e^{-ta} \prod_{i=1}^s E\Bigg[e^{tX_i}\Bigg]
\end{align*}
by independence.
\qed
\end{lemma}

\begin{theorem}Hoeffding--Chernoff Inequality for matrix-valued random variables \cite{kannan2017randomized}
	
	Let $X$ be a random variable taking values which are real symmetric $d \times d$ matrices. Suppose $X_1, X_2, \cdots , X_s$ are i.i.d. draws of $X$. For any positive real numbers $a$, $t$, we have
	
	\begin{align}
		\label{thm:chern-mat-eigen}
		\Pr(\lambda_{\max}\Big(\sum_{i=1}^s X_i\Big) \geq a ) &\leq de^{-ta} \Vert E[e^{tX}]\Vert_2^s \\
		\label{thm:chern-mat-norm}
		\Pr(\Big\Vert \Big(\sum_{i=1}^s X_i\Big)\Big\Vert_2 \geq a ) &\leq de^{-ta} (\Vert E[e^{tX}]\Vert_2^s + \Vert E[e^{-tX}]\Vert_2^s)
	\end{align}
	
	where $\lambda_{\max}$ is the largest eigenvalue.
	\begin{proof}
		First, we can show that (\ref{thm:chern-mat-eigen}) $\implies$ (\ref{thm:chern-mat-norm}). By definition of the 2-norm of a matrix,
		
		\begin{align*}
		\Vert \sum_i X_i \Vert_2 = \max\Big(\lambda_{\max} \Big(\sum_i X_i\Big), \lambda_{\max} \Big(\sum_i (-X_i)\Big)\Big)	
		\end{align*}
		
		since it is the square root of the maximum eigenvalue of $(\sum_i X_i^T) \sum_i X_i = (\sum_i X_i) \sum_i X_i$ and hence, equivalently, the maximum absolute value of an eigenvalue of $X_i$. Therefore, we can simply apply (\ref{thm:chern-mat-eigen}) to both $X_i$ and $-X_i$ and we get (\ref{thm:chern-mat-norm}).
		
		So, we can focus our attention on (\ref{thm:chern-mat-norm}). Let $S = \sum_i^s X_i$. Hence,
		
		\begin{align*}
		\lambda_{\max}(S) \geq a \iff 	\lambda_{\max}(tS) \geq ta
		\intertext{Furthermore, by considering the power series definition of the exponential,}
		\iff \lambda_{\max}(e^{tS}) \geq e^{ta}\\
		\implies \Tr(e^{tS}) \geq e^{ta}
		\end{align*}
		
since the trace is the sum of the matrix's eigenvalues. Since $\Tr(e^{tS}) \geq 0$, we can apply Markov's inequality

\begin{align*}
\Pr(\Tr(e^{tS}) \geq e^{ta}) \leq \frac{E[\Tr(e^{tS})]}{e^{ta}}
\end{align*}

Now, we use the following lemma

\begin{lemma}
Golden-Thompson Inequality

If $A$ and $B$ are Hermitian matrices, then

\begin{align*}
\Tr(e^{A + B}) \leq \Tr(e^A e^B)
\end{align*}
\qed
\end{lemma}

Hence, we can let $A = t(\sum_i^{s-1} X_i)$ and $B = tX_s$. Then,

\begin{align*}
E_X\Big[\Tr(e^{tS})\Big] &\leq E_X\Big[\Tr(e^{t\big(\sum_i^{s-1} X_i\big)}e^{tX_s})\Big]\\
\shortintertext{Since the expectation operator commutes with the summation of the trace by linearity of trace,}
&= \Tr\Big(E_X\Big[e^{t\big(\sum_i^{s-1} X_i\big)}e^{tX_s}\Big]\Big)\\
&= \Tr\Big(E_{X_1, X_2, \cdots, X_{s-1}}\Big[e^{t\big(\sum_i^{s-1} X_i\big)}\Big]E_{X_s}\Big[e^{tX_s}\Big]\Big) \tag{by independence}\\
\end{align*}

Now, we can apply Corollary (\ref{cor:psd-tr-norm-ineq}), which gives 

\begin{align*}
&\leq \Tr\Big(E_{X_1, X_2, \cdots, X_{s-1}}\Big[e^{t\big(\sum_i^{s-1} X_i\big)}\Big]\Big) \Big\Vert E_{X_s}\Big[e^{tX_s}\Big]\Big\Vert_2\\
&= \Tr\Big(E_{X}\Big[e^{t\big(\sum_i^{s-1} X_i\big)}\Big]\Big) \Big\Vert E_{X}\Big[e^{tX}\Big]\Big\Vert_2 \\
&= E_X\Big[\Tr\Big(e^{t\big(\sum_i^{s-1} X_i\big)}\Big)\Big] \Big\Vert E_{X}\Big[e^{tX}\Big]\Big\Vert_2  \intertext{So we can repeat this process iteratively, peeling an $X_i$ each time from the left term. For clarity, the next step gives,}
E_X\Big[\Tr\Big(e^{t\big(\sum_i^{s-1} X_i\big)}\Big)\Big] &\leq E_X\Big[\Tr(e^{t\big(\sum_i^{s-2} X_i\big)}e^{tX_{s-1}})\Big]\\
&\leq E_X\Big[\Tr\Big(e^{t\big(\sum_i^{s-2} X_i\big)}\Big)\Big] \Big\Vert E_{X}\Big[e^{tX}\Big]\Big\Vert_2 \tag{applying (\ref{cor:psd-tr-norm-ineq}) again}
\intertext{Therefore, after peeling all terms but the last $X_i$, we have}
E_X\Big[\Tr(e^{tS})\Big] &\leq E_X\Big[\Tr\Big(e^{tX}\Big)\Big] \Big\Vert E_{X}\Big[e^{tX}\Big]\Big\Vert_2^{s-1} \intertext{Hence, since the trace is the sum of eigenvalues, $\Tr(e^{tX}) \leq d \lambda_{\max}(e^{tX})$ i.e. the worst case of all $d$ eigenvalues being the max}
&\leq d \Big\Vert E_{X}\Big[e^{tX}\Big]\Big\Vert_2^{s}
\end{align*}

as desired.
\end{proof}
\end{theorem}

\begin{lemma}
\label{lemma:exp-eigen-approx}
If $B \in \CC^{d \times d}$ is a hermitian matrix for which $\Vert B \Vert_2 \leq 1$, then $e^{B} \leq I + B + B^2$

\begin{proof}
	We know that $e^{\lambda_i} \leq 1 + \lambda_i + \lambda_i^2$, $|\lambda_i|^2 \leq 1$. Hence,
	
	\begin{align*}
	e^{\lambda_i} \ket{v_i}\bra{v_i} \leq (1 + \lambda_i + \lambda_i^2) 	\ket{v_i}\bra{v_i}
	\end{align*}

	where $\ket{v_i}$ is the corresponding eigenvector. This then implies
	
	\begin{align*}
	e^B &= \sum_{i=1}^d e^{\lambda_i} \ket{v_i}\bra{v_i} \preceq \sum_{i=1}^d (1 + \lambda_i + \lambda_i^2) \ket{v_i}\bra{v_i} \\
	&= I + B + B^2
	\end{align*}
\end{proof}
\end{lemma}

\end{subappendices}

\end{document}